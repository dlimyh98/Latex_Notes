\documentclass{article}
\usepackage[utf8]{inputenc}
\usepackage{graphicx}
\usepackage[fleqn, leqno]{mathtools} 
\usepackage{textcomp}
\usepackage{titling}
\usepackage{subfig}
\usepackage[parfill]{parskip}
\usepackage{xcolor}
\definecolor{LightGray}{gray}{0.9}
\usepackage{titlesec}
\setcounter{secnumdepth}{4}
\usepackage[a4paper,left=1cm,right=1cm,top=1cm,bottom=1.5cm,]{geometry}
\usepackage{eqparbox}
\usepackage{enumitem}
\usepackage{relsize}
\usepackage{dsfont}
\usepackage{hyperref}

\newcommand*{\vertbar}{\rule[-1ex]{0.5pt}{2.5ex}}
\newcommand*{\horzbar}{\rule[.5ex]{2.5ex}{0.5pt}}

\makeatletter
\newcommand{\leqnomode}{\tagsleft@true\let\veqno\@@leqno}
\newcommand{\reqnomode}{\tagsleft@false\let\veqno\@@eqno}
\makeatother

\counterwithin*{equation}{section}
\counterwithin*{equation}{subsection}
\counterwithin*{equation}{subsubsection}

\title{\vspace{-2cm} Jenkins Guide for Dummies}
\date{\vspace{-5ex}}

\begin{document}
\maketitle

\section{Jenkins on Windows running WSL2}
Note that WSL2 \textit{should} already have Java installed. There is no need to install Java again on the WSL2 side.

We can download the \textit{Jenkins Generic Java Package}, and run it using the \textbf{modified} command \textit{java.exe -jar jenkinswar --httpPort=8080}

We will be setting up Jenkins using the classical UI method (no Docker images for us)

\subsection{Error connecting to repo}
Manage>Configure Tools>Git Installations>Path to Git executable

Type \textit{which git} to find path of git executable

Then, access on WSL2 via \textit{\textbackslash\textbackslash wsl\textbackslash Ubuntu} and navigate to that file

\end{document}